\chapter{Conclusion} \label{chap:conclusion}
\section{Overview}
This project highlights the debate concerning the efficient market hypothesis and the feasability
of predicting stock markets, particularly the US stock market. This debate is likely to continue 
for many years to come, but the project shows some evidence of market inefficiencies in short term
horizons that neural networks can detect from both short-term and mid-term sequence lengths.
This project looks at the performance of a few neural network models including Convolutional Neural
Networks (CNNs), Long-Short Term Memory networks (LSTMs) and hybrid approachs (CNN-LSTMs).

The artefact generated in this project has a meaningful accuracy in stock market forecasting,
and the results from final iteration have been able to aid in optimal feature selection in terms
of input features and sequence lengths. The results of the models can be used to aid investors'
decisions in what factors they focus on when making investment decisions as well as to allow
investors to potentially take advantage of market inefficiencies.

\section{Future Work}
For the future, additional input features or different combinations of input features can be tested
on the same research method to identify if there are opportunities for further optimisations.
Furthermore, the same input features can be tested on different research models to verify the results
of the optimal feature selection identified in this project. With an increased amount of time, or with
more powerful hardware, additional neural network parameters such as the layer sizes and amounts can be
tested to identify if there are better performing models.

While this project was not able to identify specific future prices of the stock market, that is an
area that can be looked into further utilising the factors identified in this project. This could be
helpful specifically to the options market where specific prices are important to participants.

Additionally, as this project identifies some input features are more or less important depending on
the sequence length of historical data; an approach to exploit the advantages of different
sequence lengths could prove to produce a model with greater accuracy.

Moreover, long periods of time may cause patterns recognised by the neural network in the training data
set to be outdated in the validation dataset. There should be further work done in identifying the maximum
amount of history to use in order to avoid networks recognising outdated patterns, whilst ensuring the
network has enough history to learn relevant patterns.
