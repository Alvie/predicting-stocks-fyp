\chapter{Research Design} \label{chap:research-design}
\section{Research method}
The research methodology is one based on the literature review (\autoref{chap:literature-review});
the research already carried out in other studies
informed the the decision making process behind the requirements of the supporting artefact. Various artificial
intelligence models have been analysed and critiqued; as well as the input features used within these models.

These aspects combined are used to inform the variables - such as model used, input features
(and their time horizons) - to experiment with to better understand which variables are most important
and have the ability to provide the most accurate results.


\section{Research process}
\subsection{Comparison of AI models in previous studies}
From the literature review, various AI models were identified. These include Multilayer Perceptrons (MLPs),
Convolutional Neural Networks (CNNs), Support Vector Machines (SVMs), Long Short Term Memory (LSTM) and hybrid
approaches. From this, we can understand which models can have greater accuracy. The review has found a hybrid
approach has the greatest accuracy. CNNs and SVMs have a similar degree of accuracy and generally outperform MLPs.

\begin{table}[h]
    \centering
    \begin{tabular}{|c|c|c|}
        \hline
        Artificial Intelligence Model & Ease of Implementation & Accuracy of Output \\
        \hline\hline
        Multilayer Perceptrons & 5 & 3 \\
        Convolutional Neural Networks & 3 & 4 \\
        Support Vector Machines & 3 & 4 \\
        Hybrid (CNN + LSTM) & 2 & 5 \\
        \hline
    \end{tabular}
    \caption[Table rating different AI models]{Table rating different AI models on a scale of 1 to 5 based on ease of implementation as well as accuracy}
    \label{tab:research_model_comparison}
\end{table}
\FloatBarrier

\subsection{Comparison of input features used in previous studies}
Currently, based on the studies in the literature review, it is difficult to suggest which input features
are most important for an accurate output. This is due to the studies using more complex datasets as input
features generally use a different research model 

This is one of the primary research questions identified, and this project intends to answer this question.
The previous studies use the following input features; a subset of the features will be used and compared
within this project. 
\begin{itemize}
    \item Daily Closing Price
    \item Low Price
    \item High Price
    \item Volume
    \item Currency Rates
    \item Commodities (e.g. Oil)
    \item Treasury Rates
    \item Stock indices of other countries
    \item Individual Stocks
    \item Certificate of Deposits
    \item Term spreads
\end{itemize}

\section{Research outcomes}
As CNN+LSTM hybrid approaches have already been proven in previous studies to be an accurate research model,
this model should be used within the artefact of this project. However, the effect of different input
features on a single research model has not been well identified. This will be a primary area of focus for
the supporting artefact to look into. Ideally, this will allow the supporting artefact to have greater than
65\% accuracy in predicting the daily price direction.

\section{Requirements of the supporting artefact}
\subsection{Functional requirements}
\subsubsection{Necessary requirements}
\begin{itemize}
    \item Predict daily price direction with 65\%+ accuracy
    \item 
\end{itemize}
\subsubsection{Optional requirements}
\begin{itemize}
    \item Predict daily price direction with 65\%+ accuracy
    \item 
\end{itemize}
\subsection{Non-functional requirements}
\begin{itemize}
    \item Predict daily price direction with 65\%+ accuracy
    \item 
\end{itemize}