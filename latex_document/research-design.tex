\chapter{Research Design} \label{chap:research-design}
\section{Research method}
The research methodology is one based on the literature review (\autoref{chap:literature-review});
the research already carried out in other studies
informed the the decision making process behind the requirements of the supporting artefact. Various artificial
intelligence models have been analysed and critiqued; as well as the input features used within these models.

These aspects combined are used to inform the variables - such as model used, input features
(and their time horizons) - to experiment with to better understand which variables are most important
and have the ability to provide the most accurate results.


\section{Research process}
\subsection{Comparison of AI models in previous studies}
From the literature review, various AI models were identified. These include Multilayer Perceptrons (MLPs),
Convolutional Neural Networks (CNNs), Support Vector Machines (SVMs), Long Short Term Memory (LSTM) and hybrid
approaches. From this, we can understand which models can have greater accuracy. The review has found a hybrid
approach has the greatest accuracy. CNNs and SVMs have a similar degree of accuracy and generally outperform MLPs.

\begin{table}[h]
    \centering
    \begin{tabular}{|c|c|c|}
        \hline
        Artificial Intelligence Model & Ease of Implementation & Accuracy of Output \\
        \hline\hline
        Multilayer Perceptrons & 5 & 3 \\
        Convolutional Neural Networks & 3 & 4 \\
        Support Vector Machines & 3 & 4 \\
        Hybrid (CNN + LSTM) & 2 & 5 \\
        \hline
    \end{tabular}
    \caption[Table rating different AI models]{Table rating different AI models on a scale of 1 to 5 based on ease of implementation as well as accuracy}
    \label{tab:research_model_comparison}
\end{table}
\FloatBarrier

\subsection{Comparison of input features used in previous studies}
Currently, based on the studies in the literature review, it is difficult to suggest which input features
are most important for an accurate output. This is due to the studies using more complex datasets as input
features generally use a different research model 

This is one of the primary research questions identified, and this project intends to answer this question.
The previous studies use the following input features; a subset of the features will be used and compared
within this project. 
\begin{itemize}
    \item Daily Closing Price
    \item Low Price
    \item High Price
    \item Volume
    \item Currency Rates
    \item Commodities (e.g. Oil)
    \item Treasury Rates
    \item Stock indices of other countries
    \item Individual Stocks
    \item Certificate of Deposits
    \item Term spreads
\end{itemize}

\section{Research outcomes}
As CNN+LSTM hybrid approaches have already been proven in previous studies to be an accurate research model,
this model should be used within the artefact of this project. However, the effect of different input
features on a single research model has not been well identified. This will be a primary area of focus for
the supporting artefact to look into. Ideally, this will allow the supporting artefact to have greater than
65\% accuracy in predicting the daily price direction.

\section{Requirements of the artefact}
\subsection{Functional requirements}
Based on the literature review, the most accurate models for predictting stock prices had an accuracy of
approximately 60\% \parencite{zhong2019predicting}. While there is an example of a study shown with a greater
accuracy of 76\% \parencite{shen2012stock}, another similar study places the accuracy of the model at 59.3\%
- albeit for weekly predictions rather than daily \parencite{hao_gao_2020}; thus 60\% is considered the
current baseline.\\
Also, due to the fact that this research project attempts to identify the most important input features, as
well as the sequence length, they are key requirements of this project.\\
Predicting the exact return (daily relative price change) per day is an optional requirement due to two
reasons: it does not significantly help market participants compared to knowing the direction alone; and
it may be more cumbersome / time-consuming for a relatively low significant output for end users.
\subsubsection{Necessary requirements}
\begin{itemize}
    \item Predict daily price direction with 65\%+ accuracy
    \item Identify which input features are most important
    \item Identify what sequence length (number of days of history of each input feature) is optimal
    \item Display charts of the various models' accuracies to allow user to visualise and compare each model
\end{itemize}
\subsubsection{Optional requirements}
Based on the literature review, a study suggested it had been able to accurately predict prices with a mean
absolute deviation of 2.516\% on the testing data set when compared to the true price.
\begin{itemize}
    \item Predict daily price return with a mean absolute deviation of 2.5\% or lower
\end{itemize}
\subsection{Non-functional requirements}
The performance artefact is not time critical due to there being a large time period 17 hours and 30 minutes
between market close (4:00 PM ET) and market open (9:30 AM ET) of stock exchanges in the USA; however this
may not be ideal for users thus the following non-functional requirements have been identified.
\begin{itemize}
    \item The artefact should take less than 5 minutes to run per model chosen (on modern GPUs)
    \item The artefact should allow the user to choose which input features to include
    with less than 1 click per feature + 1 for entering selection window
    \item The artefact should allow the user to choose the sequence lengths to include
    with less than 1 click per sequence length + 1 for entering selection window
\end{itemize}