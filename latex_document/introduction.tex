\chapter{Introduction} \label{chap:introduction}

\section{Overview}

Stock markets are often thought to be irrational or in some cases random,
which suggests that markets are not made based on fundamental factors alone.
In a 1996 speech, former Federal Reserve chair, Alan Greenspan questioned ``... how do we know when irrational exuberance has unduly escalated asset
values ...'' \parencite{greenspan1996}, suggesting that there may be
psychological factors that affect markets such as positive feedback loops.\\

This project intends to identify if a stock market index's price can be predicted ahead
of time, and which factors contribute the most to an accurate prediction.
The intended audience for such a project can vary from the institutional to
the retail investor to identify opportunities in the market by using only
publicly available data rather than proprietary information.

\section{Aims}

This project aims to produce an artificial intelligence model that
attempts to predict the following day's price of a stock market index. This will
allow the project to research if a stock market index's price can be predicted
from the day prior; the factors which affect a stock market index's price and which
factors are most important in predicting a stock market index's price.

\section{Objectives}

The objectives are as follows:
\begin{itemize}
    \item Identifying key factors that can affect stock market performance.
    \item Collecting above mentioned factors, where the historical data is publicly available and free to use (either public domain, non-commercial licenses, etc).
    \item Transforming collected data into normalised, or otherwise modified, data that can easily be used to identify patterns, either by humans or by computational neural networks.
    \item Identifying and applying correct techniques to train a computational neural network (or other suitable technology) to understand which factors are most important when predicting stock market performance.
    \item Filtering through training data to ensure there are no biases in play.
    \item Ensuring that the network will be efficient enough to be trained on hardware that is available today, and if possible on consumer hardware such as Nvidia RTX 3000 series GPUs.
    \item Understanding outputs of the computational neural networks and presenting the output data learned in a way that is clear.
    \item Interpreting overall results to identify if it is possible to predict a stock market's index direction and/or price for any given day
    \item Evauluating the results to identify accuracy, limitations and biases
\end{itemize}

\section{Constraints}
\begin{itemize}
    \item May be difficult / expensive to get historical data
    \item May be difficult / expensive to get reliable data
    \item Access to currently publicly available data may be removed, or otherwise unavailable
    \item May not have the correct technical skills to sufficiently carry out the projects
    \item May not have sufficient time to carry out the full objectives of the project
    \item Hardware available may not be feasible to run / train complex neural networks on
\end{itemize}

Even if the project was successful for now, it may not be suitable for these reasons:
\begin{itemize}
    \item May not work in extreme market events
    \item May not work if new market measures are placed
    \item May not work if new economic measures are placed
    \item Access to reliable data may be restricted, removed or otherwise unavailable in the future
\end{itemize}
